% !TEX TS-program = pdflatex
% !TEX encoding = UTF-8 Unicode

% This is a simple template for a LaTeX document using the "article" class.
% See "book", "report", "letter" for other types of document.

\documentclass[a4paper, 12pt]{article} % use larger type; default would be 10pt

\usepackage[utf8]{inputenc} % set input encoding (not needed with XeLaTeX)

%%% Examples of Article customizations
% These packages are optional, depending whether you want the features they provide.
% See the LaTeX Companion or other references for full information.

%%% PAGE DIMENSIONS
\usepackage{geometry} % to change the page dimensions
\geometry{a4paper} % or letterpaper (US) or a5paper or....
\geometry{margin=2cm} % for example, change the margins to 2 inches all round
% \geometry{landscape} % set up the page for landscape
%   read geometry.pdf for detailed page layout information

\usepackage{graphicx} % support the \includegraphics command and options

% \usepackage[parfill]{parskip} % Activate to begin paragraphs with an empty line rather than an indent

%%% PACKAGES
\usepackage{booktabs} % for much better looking tables
\usepackage{array} % for better arrays (eg matrices) in maths
\usepackage{paralist} % very flexible & customisable lists (eg. enumerate/itemize, etc.)
\usepackage{verbatim} % adds environment for commenting out blocks of text & for better verbatim
\usepackage{subfig} % make it possible to include more than one captioned figure/table in a single float
% These packages are all incorporated in the memoir class to one degree or another...

%%% HEADERS & FOOTERS
\usepackage{fancyhdr} % This should be set AFTER setting up the page geometry
\pagestyle{fancy} % options: empty , plain , fancy
\renewcommand{\headrulewidth}{0pt} % customise the layout...
\lhead{}\chead{}\rhead{}
\lfoot{}\cfoot{\thepage}\rfoot{}

%%% SECTION TITLE APPEARANCE
\usepackage{sectsty}
\allsectionsfont{\sffamily\mdseries\upshape} % (See the fntguide.pdf for font help)
% (This matches ConTeXt defaults)

%%% ToC (table of contents) APPEARANCE
\usepackage[nottoc,notlof,notlot]{tocbibind} % Put the bibliography in the ToC
\usepackage[titles,subfigure]{tocloft} % Alter the style of the Table of Contents
\renewcommand{\cftsecfont}{\rmfamily\mdseries\upshape}
\renewcommand{\cftsecpagefont}{\rmfamily\mdseries\upshape} % No bold!

\usepackage {stmaryrd}
\newcommand{\eqgood}{\overset{\surd}{=\joinrel=}}
\newcommand{\eqbad}{{=\hspace{-1.3ex} \lightning \hspace{-1.3ex} \joinrel =}}

\usepackage[german]{babel}
\usepackage{fontenc}
\usepackage{multirow}
\usepackage{amsmath}
\usepackage {relsize}
\usepackage{diagbox}
\usepackage{tikz}
\usepackage{Mnsymbol}
\tikzset{
  treenode/.style = {shape=rectangle, rounded corners,
                     draw, align=center,
                     top color=white, bottom color=blue!20},
  root/.style     = {treenode, font=\Large, bottom color=red!30},
  env/.style      = {treenode, font=\ttfamily\normalsize},
  dummy/.style    = {circle,draw},
  leaf/.style    = {treenode, bottom color=green!30}
}

\linespread{1.2}
%%% END Article customizations

%%% The "real" document content comes below...

%\date{} % Activate to display a given date or no date (if empty),
         % otherwise the current date is printed 

\newcommand{\aufgabe}[1]{{\huge Statistik Übung \underline{Aufgabe #1}}\\[3.5ex]  } 

\begin{document}
\aufgabe{15} 
\begin{tabular}{| c || c | c | c || c | c | c | c | c |}
\hline
\diagbox{$Y$}{$X$}
&
$0$
&
$2$
&
$4$
&
$y$
&
$P(Y=y)$
&
$y * P(Y = y)$
&
$y^2$
&
$y^2 * P(Y = y)$
\\
\hline
\hline
$-1$
&
$\frac{8}{32}$
&
$\frac{1}{32}$
&
$\frac{1}{32}$
&
$-1$
&
$\frac{10}{32}$
&
$-\frac{10}{32}$
&
1
&
$\frac{10}{32}$
\\
\hline
$0$
&
$\frac{4}{32}$
&
$\frac{5}{32}$
&
$0$
&
$0$
&
$\frac{9}{32}$
&
$0$
&
$0$
&
$0$
\\
\hline
$2$
&
$\frac{4}{32}$
&
$\frac{8}{32}$
&
$\frac{1}{32}$
&
$2$
&
$\frac{13}{32}$
&
$\frac{26}{32}$
&
4
&
$\frac{52}{32}$
\\
\hline
\hline
$x$
&
$0$
&
$2$
&
$4$
&
\multirow{2}*{}
&
\multirow{2}*{$\sum = \frac{32}{32}$}
&
\multirow{2}*{$E(Y) = \frac{16}{32}$}
&
\multirow{2}*{}
&
\multirow{2}*{$E(Y^2) = \frac{62}{32}$}
\\
\cline{1-4}
$P(X = x)$
&
$\frac{16}{32}$
&
$\frac{14}{32}$
&
$\frac{2}{32}$
&
&
&
&
&
\\
\hline
$x * P(X=x)$
&
$0$
&
$\frac{28}{32}$
&
$\frac{8}{32}$
& 
\multicolumn{5}{ l|}{$E(X) = \frac{36}{32}$}
\\
\hline
$x^2$
&
$0$
& 
$4$
&
$16$
&
\multicolumn{5}{ l|}{}
\\
\hline
$x^2 * P(X=x)$
&
$0$
&
$\frac{56}{32}$
&
$\frac{32}{32}$
&
\multicolumn{5}{ l|}{$E(X^2) = \frac{88}{32}$}
\\
\hline
\end{tabular}\\[1ex]
\begin{tabular}{| c || c | c | c || c |}
\hline
$Z = min(X, Y)$ 
& 
$-1$
&
$0$
&
$2$
&
$\sum$
\\
\hline
$P(Z = z)$
&
$\frac{10}{32}$
&
$\frac{13}{32}$
&
$\frac{9}{32}$
&
$\frac{32}{32}$
\\
\hline
$z * P(Z = z)$
&
$-\frac{10}{32}$
&
$0$
&
$\frac{18}{32}$
&
$E(Z) = \frac{8}{32}$
\\
\hline
$Z^2$
&
$1$
&
$0$
&
$4$
&
\\
\hline
$z^2 * P(Z = z)$
&
$\frac{10}{32}$
&
$0$
&
$\frac{36}{32}$
&
$E(Z^2) = \frac{46}{32}$
\\
\hline
\end{tabular}\\[1ex]
\begin{tabular}{ | c || c | c | c || c | }
\hline
$ZY=w$
&
$0$
&
$1$
& 
$4$
&
$\sum$
\\
\hline
Urbilder 
&
$\{Y = 0\} \cupdot \{Y=2 \wedge X=0\}$
&
$\{Y = -1 \}$
& 
$\{Y = 2 \wedge X \neq 0\}$
& 
$\Omega$
\\
\hline
$P(ZY= w)$
&
$\frac{13}{32}$
&
$\frac{10}{32}$
&
$\frac{9}{32}$
& 
$\frac{32}{32}$
\\
\hline
$w * P(ZY= w)$
&
$0$
&
$\frac{10}{32}$
&
$\frac{36}{32}$
& 
$E(ZY) = \frac{46}{32}$
\\
\hline
\end{tabular}\\[1ex]
\begin{tabular}{l l l}
$E(X) = \frac{36}{32} = \frac{9}{8} = 1,125 $ 
&&
$Var(X) = \frac{88}{32} - (\frac{36}{32})^2 = 1,484375 \approx 1,5 $ 
\\
$E(Y) = \frac{16}{32} = \frac{1}{2} = 0,5 $ 
&\hspace{2cm} \; &
$Var(Y) = \frac{62}{32} - (\frac{1}{2})^2 = 1,6875 \approx 1,7 $ 
\\
$E(Z) = \frac{8}{32} = \frac{1}{4} = 0,25 $ 
&&
$Var(Z) = \frac{46}{32} - (\frac{1}{4})^2 = 1,375 $ 
\\
\multicolumn{3}{l }{$CoVar(Z, Y) = E(ZY) - E(Z) E(Y) = \frac{46}{32} - \frac{1 * 1}{4 * 2 } =  1,3125$}
\\
\end{tabular}\\[1ex]
\- \dotfill
\\[4ex]
\aufgabe{16}
\begin{tabular}{ | c | | c | c | c | c || c | }
\hline
\diagbox{$X$}{$Y$}
&
kein
&
gering
&
mittel
&
groß
&
$\sum$
\\
\hline
\hline
Ulme
&
$8$
&
$6$
&
$5$
&
$1$
&
$20$
\\
\hline
Kiefer 
&
$7$
&
$10$
&
$3$
&
$0$
&
$20$
\\
\hline
 Fichte
&
$2$
&
$6$
&
$15$
&
$8$
&
$31$
\\
\hline
\hline
$\sum$
&
$17$
&
$22$
&
$23$
&
$9$
&
$71$
\\
\hline
\hline
Ulme
&
$40\%$
&
$30\%$
&
$25\%$
&
$5\%$
& 
$100\%$
\\
\hline
\end{tabular}\\[1ex]
$X$ und $Y$ sind nicht unabhängig, weil \\$P(X= $ Kiefer$, Y = $ groß$) = 0 \; \eqbad \; \frac{20 * 9}{71 * 71} = P(X=$ Kiefer$) * P(Y = $ groß$)$
\newpage
\aufgabe{17}
\begin{figure}[htbp]
\begin{tikzpicture} [
 grow = right, 
 level 1/.style={sibling distance=16em},    
 level 2/.style={sibling distance=7em}, 
 level 3/.style={sibling distance=3em}, 
 level 4/.style={sibling distance=3em}, 
 level distance = 9em, 
 edge from parent/.style = {draw, -latex},
 every node/.style       = {font=\footnotesize},
 sloped	 ]
\node [root] {}
child {
  node [env] {$G$}
  child {
    node [env] {$G$}
    child {
      node [env] {$G$} 
        child {
          node [leaf] {$P(GGG) = \frac{1 * 1 * 1}{7^3} $} [white]
        }
       edge from parent node [below] {$1/7$}          
    }
    child {
      node [env] {$B$}
        child {
          node [leaf] {$P(GGB) = \frac{6 * 1 * 1}{7^3} $} [white]
        }
       edge from parent node [above] {$6/7$}          
    }
     edge from parent node [above] {$1/7$}          
  }
  child {
    node [env] {$B$}
    child {
      node [env] {$G$} 
        child {
          node [leaf] {$P(GBG) = \frac{6 * 2 * 1}{7^3} $} [white]
        }
       edge from parent node [below] {$2/7$}          
    }
    child {
      node [env] {$B$}
        child {
          node [leaf] {$P(GBB) = \frac{6 * 5 * 1}{7^3} $} [white]
        }
       edge from parent node [above] {$5/7$}          
    }
     edge from parent node [above] {$6/7$}          
  }
   edge from parent node [above] {$1/7$}          
}
child {
  node [env] {$B$}
  child {
    node [env] {$G$}
    child {
      node [env] {$G$} 
        child {
          node [leaf] {$P(BGG) = \frac{6 * 2 * 2}{7^3} $} [white]
        }
       edge from parent node [below] {$2/7$}          
    }
    child {
      node [env] {$B$}
        child {
          node [leaf] {$P(BGB) = \frac{6 * 5 * 2}{7^3} $} [white]
        }
       edge from parent node [above] {$5/7$}          
    }
     edge from parent node [above] {$2/7$}          
  }
  child {
    node [env] {$B$}
    child {
      node [env] {$G$} 
        child {
          node [leaf] {$P(BBG) = \frac{6 * 5 * 3}{7^3} $} [white]
        }
       edge from parent node [below] {$3/7$}          
    }
    child {
      node [env] {$B$}
        child {
          node [leaf] {$P(BBB) = \frac{6 * 5 * 4}{7^3} $} [white]
        }
       edge from parent node [above] {$4/7$}          
    }
     edge from parent node [above] {$5/7$}          
  }
   edge from parent node [above] {$6/7$}          
};
\end{tikzpicture}
\end{figure}
\\[1ex]
Sei $X$ die Anzahl der gezogenen gelben Karten und $Y$ der Auszahlungsbetrag des Spielers. \\
\begin{tabular}{|c||c|c|c|c||c|}
\hline 
$X$	&		$0$	&	$1$	&	$2$	&	$3$	& 	$\sum$	\\ \hline
\hline
$7^3 * P(X=x)$ &	$120$	&	$180$	&	$42$	&	$1$	&	$343$	\\ \hline
$Y$	 &		$0$	&	$2$	&	$5$	&	$100$	&		\\ \hline
$y * P(Y=y)$ &	$0$	&	$\frac{360}{343}$ & $\frac{210}{343}$ & $\frac{100}{343}$	& 	$E(Y) = \frac{670}{343} \approx 1,95$ \\ \hline
\end{tabular}\\[1ex]
Mit einer Teilnahmegebühr von {\boldmath$2,95 $ Euro} macht die Bank \\im Schnitt pro Spiel einen Euro Gewinn. \\
\newpage
\aufgabe{Ü 3.1}
Sei $W$ das Ergebnis des in der Aufgabe beschriebenen Experiments. \\
\begin{tabular}{| c || c | c | c | c | c | c || c | }
\hline
$W$ 	& 	1 	& 	2 	& 	3 	& 	4 	& 	5 	& 	6 	& $\sum$ \\ \hline
$P(W=w)$ &	$\frac{5}{18}$ &	$\frac{5}{18}$ &	$\frac{5}{18}$ &	$\frac{1}{18}$ &	$\frac{1}{18}$ &	$\frac{1}{18}$ &	$\frac{18}{18}$  \\ \hline
\end{tabular} \\[2cm]
\- \dotfill
\\[4ex]
\aufgabe{Ü 3.2}
Die $6$ Plätze in Fahrtrichtung auf $5$ Personen verteilen: $6! / 1! = 720$ Möglichkeiten \\
Die $6$ Plätze gegen Fahrtrichtung auf $4$ Personen verteilen: $6! / 2! = 360$ Möglichkeiten \\
Die übrigen $3$ Plätze auf die übrigen $3$ Personen verteilen: $3! = 6$ Möglichkeiten \\
Insgesamt gibt es $720 * 360 * 6 = 1.555.200$ Möglichkeiten, die Gruppe zu verteilen \\
\bigskip
Sieht man sich nur die Abteile an: \\
Wir betrachten nur Abteil 1, weil alle, die nicht dort sitzen, sind automatisch in Abteil 2. \\
Sei $F_{VW}$ die Menge der 5 Personen, die nur vorwärts fahren wollen. \\
Sei $F_{RW}$ die Menge der 4 Personen, die nur rückwärts fahren wollen. \\
Sei $F_{F}$ die Menge der 3 Personen, denen die Richtung egal ist. \\
Zu den 3 Mengen seien $F_{VW}^1$, $F_{RW}^1$ und $F_{F}^1$ die Teilmengen, die in Abteil 1 sitzen. \\[1ex]
\begin{tabular}{|c|c|c|cc|}
\hline
$\#F_{VW}^1$ 	&	$\#F_{RW}^1$ 	&	$\#F_{F}^1$ 	&	Möglichkeiten 	& \\ \hline \hline
$2$ & $1$ & $3$ & $10*4*1 = $ & $40$\\ \hline
$2$ & $2$ & $2$ & $10*6*3 = $ & $180$\\ \hline
$2$ & $3$ & $1$ & $10*4*3 = $ & $120$\\ \hline
$3$ & $1$ & $2$ & $10*4*3 = $ & $120$\\ \hline
$3$ & $2$ & $1$ & $10*6*3 = $ & $180$\\ \hline
$3$ & $3$ & $0$ & $10*4*1 = $ & $40$\\ \hline \hline
\multicolumn{3}{|r|}{$\sum$} & & $680$ \\ \hline
\end{tabular}
\end{document}