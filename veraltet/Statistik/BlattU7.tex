% !TEX TS-program = pdflatex
% !TEX encoding = UTF-8 Unicode

% This is a simple template for a LaTeX document using the "article" class.
% See "book", "report", "letter" for other types of document.

\documentclass[12pt]{article} % use larger type; default would be 10pt

\usepackage[utf8]{inputenc} % set input encoding (not needed with XeLaTeX)

%%% Examples of Article customizations
% These packages are optional, depending whether you want the features they provide.
% See the LaTeX Companion or other references for full information.

%%% PAGE DIMENSIONS
\usepackage{geometry} % to change the page dimensions
\geometry{a4paper} % or letterpaper (US) or a5paper or....
\geometry{margin=1cm} % for example, change the margins to 2 inches all round
% \geometry{landscape} % set up the page for landscape
%   read geometry.pdf for detailed page layout information

\usepackage{graphicx} % support the \includegraphics command and options

% \usepackage[parfill]{parskip} % Activate to begin paragraphs with an empty line rather than an indent

%%% PACKAGES
\usepackage{booktabs} % for much better looking tables
\usepackage{array} % for better arrays (eg matrices) in maths
\usepackage{paralist} % very flexible & customisable lists (eg. enumerate/itemize, etc.)
\usepackage{verbatim} % adds environment for commenting out blocks of text & for better verbatim
\usepackage{subfig} % make it possible to include more than one captioned figure/table in a single float
% These packages are all incorporated in the memoir class to one degree or another...

%%% HEADERS & FOOTERS
\usepackage{fancyhdr} % This should be set AFTER setting up the page geometry
\pagestyle{fancy} % options: empty , plain , fancy
\renewcommand{\headrulewidth}{0pt} % customise the layout...
%\lhead{}\chead{}\rhead{}
%\lfoot{}\cfoot{}\rfoot{}

%%% SECTION TITLE APPEARANCE
\usepackage{sectsty}
\allsectionsfont{\sffamily\mdseries\upshape} % (See the fntguide.pdf for font help)
% (This matches ConTeXt defaults)

%%% ToC (table of contents) APPEARANCE
\usepackage[nottoc,notlof,notlot]{tocbibind} % Put the bibliography in the ToC
\usepackage[titles,subfigure]{tocloft} % Alter the style of the Table of Contents
\renewcommand{\cftsecfont}{\rmfamily\mdseries\upshape}
\renewcommand{\cftsecpagefont}{\rmfamily\mdseries\upshape} % No bold!

%%% END Article customizations

\newcommand{\aufgabe}[1]{{\huge Statistik Übung \underline{Aufgabe #1}}\\[3.5ex]  } 
\usepackage{amsmath}
\usepackage{amssymb}
%%% The "real" document content comes below...

\begin{document}
\aufgabe{Ü7.1}
\begin{tabular}{l l}
Indikatorzufallsvariable $A$: & Der Patient hat Antikörper. \\
Indikatorzufallsvariablen $X_i$: & Labor $i$ findet Antikörper. \\
Zufallsvariable $X = \sum_{i=1}^{3}{X_i}$ & ist die Anzahl der Labore, die Antikörper finden. \\
Behauptung: & $P(A ~ | ~ X=2) = 30,88\%$ \\
Positiv wenn positiv: & $p_A := P(X_1 | A ) = 90 \%$ \\
Positiv wenn negativ: & $p_{\overline{A}} := P(X_1 | \overline{A}) = 20\%$ \\
Verbreitung: & $P(A) = 15\%$ \\
\end{tabular}\\
$P(X=2 ~ | ~ A) = B(3, p_A)({2}) = \binom{3}{2}*p_{A}^2*\overline{p_A}^1 = 24,30 \%$ \\
$P(X=2 ~ | ~ \overline{A}) = B(3, p_{\overline{A}})({2}) = \binom{3}{2}*p_{\overline{A}}^2*\overline{p_{\overline{A}}}^1 = 9,60 \%$ \\
$P(X=2) \overset{\text{\tiny{Totale Wahrscheinlichkeit}}}{=\joinrel=\joinrel=\joinrel=\joinrel=\joinrel=\joinrel=\joinrel=\joinrel=\joinrel=\joinrel=\joinrel=} P(A) * P(X=2 ~ | ~ A) + P(\overline{A}) * P(X=2 ~ | ~ \overline{A}) ~ = $ \\
$~~ = ~ 15 \% * 24,30 \% + 85 \% * 9,60 \% = 11,805\% $ \\ 
$P(A ~ | ~ X=2) \overset{\text{\tiny{Bayes}}}{=\joinrel=\joinrel=} P(X=2 ~ | ~ A) * \frac{P(A)}{P(X=2)} = 24,30\% * \frac{15\%}{11,805\%} = 30,876747\% $ \\[1ex]
\- \dotfill
\\[4ex]
\aufgabe{Ü7.2}
$X_k$ wie in Angabe, $Y_k = X_{7-k}$ ist die Wartezeit auf die $k$-letzte Zahl, \\
$Y = \sum_{k=1}^{6}{Y_k}$, $Y_k \sim \text{Geo}(\frac{k}{6})$\\ 
$P(Y_k=w) = (k/6) * (1- k/6)^{w-1}$ \\
$E(Y_k) = 6/k$ \\
$Var(Y_k) = (1-k/6) / (6/k)^2$
$Y_k$ unabhängig $ => $ unkorrelliert \\[0.1cm]
$E(Y) = \sum_{k=1}^{6}{E(Y_k)} = 14,7 $ \\
$Var(Y) = \sum_{k=1}^{6}{Var(Y_k)} = 30+6+2+0,75+0,24+0 = 38,99 $\\
$\sigma(Y) = 6,24$ \\[1ex]
\- \dotfill
\end{document}
