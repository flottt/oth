% !TEX TS-program = pdflatex
% !TEX encoding = UTF-8 Unicode

% This is a simple template for a LaTeX document using the "article" class.
% See "book", "report", "letter" for other types of document.

\documentclass[a4paper, 12pt]{article} % use larger type; default would be 10pt

\usepackage[utf8]{inputenc} % set input encoding (not needed with XeLaTeX)

%%% Examples of Article customizations
% These packages are optional, depending whether you want the features they provide.
% See the LaTeX Companion or other references for full information.

%%% PAGE DIMENSIONS
\usepackage{geometry} % to change the page dimensions
\geometry{a4paper} % or letterpaper (US) or a5paper or....
\geometry{margin=2cm} % for example, change the margins to 2 inches all round
% \geometry{landscape} % set up the page for landscape
%   read geometry.pdf for detailed page layout information

\usepackage{graphicx} % support the \includegraphics command and options

% \usepackage[parfill]{parskip} % Activate to begin paragraphs with an empty line rather than an indent

%%% PACKAGES
\usepackage{booktabs} % for much better looking tables
\usepackage{array} % for better arrays (eg matrices) in maths
\usepackage{paralist} % very flexible & customisable lists (eg. enumerate/itemize, etc.)
\usepackage{verbatim} % adds environment for commenting out blocks of text & for better verbatim
\usepackage{subfig} % make it possible to include more than one captioned figure/table in a single float
% These packages are all incorporated in the memoir class to one degree or another...

%%% HEADERS & FOOTERS
\usepackage{fancyhdr} % This should be set AFTER setting up the page geometry
\pagestyle{fancy} % options: empty , plain , fancy
\renewcommand{\headrulewidth}{0pt} % customise the layout...
\lhead{}\chead{}\rhead{}
\lfoot{}\cfoot{\thepage}\rfoot{}

%%% SECTION TITLE APPEARANCE
\usepackage{sectsty}
\allsectionsfont{\sffamily\mdseries\upshape} % (See the fntguide.pdf for font help)
% (This matches ConTeXt defaults)

%%% ToC (table of contents) APPEARANCE
\usepackage[nottoc,notlof,notlot]{tocbibind} % Put the bibliography in the ToC
\usepackage[titles,subfigure]{tocloft} % Alter the style of the Table of Contents
\renewcommand{\cftsecfont}{\rmfamily\mdseries\upshape}
\renewcommand{\cftsecpagefont}{\rmfamily\mdseries\upshape} % No bold!

\usepackage {stmaryrd}
\newcommand{\eqgood}{\overset{\surd}{=\joinrel=}}
\newcommand{\eqbad}{{=\hspace{-1.3ex} \lightning \hspace{-1.3ex} \joinrel =}}

\usepackage[german]{babel}
\usepackage{fontenc}
\usepackage{amsmath}
\usepackage {relsize}
\usepackage[thinlines]{easytable}
\usepackage{verbatim}
\usepackage{tikz}
\tikzset{
  treenode/.style = {shape=rectangle, rounded corners,
                     draw, align=center,
                     top color=white, bottom color=blue!20},
  root/.style     = {treenode, font=\Large, bottom color=red!30},
  env/.style      = {treenode, font=\ttfamily\normalsize},
  dummy/.style    = {circle,draw},
  leaf/.style    = {treenode, bottom color=green!30}
}

\lineskiplimit=-100pt\relax
\linespread{1.2}
%%% END Article customizations

%%% The "real" document content comes below...

%\date{} % Activate to display a given date or no date (if empty),
         % otherwise the current date is printed 

\newcommand{\aufgabe}[1]{{\huge Statistik Übung \underline{Aufgabe #1}}\\[3.5ex]  } 

\begin{document}
\aufgabe{11} 
Gegeben sei die Menge $M := \{1,2,3,4,5,6,7,8\}$ und \\ ein Wahrscheinlichkeitsmaß auf dieser Menge mit $P[\{i\}] = \frac{1}{8} \; \forall i \in M$. \\[1ex]
Für die Ereignisse $D := \emptyset$ und $E := M$ gilt: \\
$\surd$ \hspace{1cm}$D$ und $E$ sind \textbf{disjunkt}: $D \cap E = \emptyset$ \\
$\surd $ \hspace{1cm}$D$ und $E$ sind \textbf{unabhängig}: $0 = P(D \cap E) \overset{\surd}{=\joinrel=} P(D) * P(E) = 0 * 1$ \\
$\surd $ \hspace{1cm}$D$ und $E$ ergeben sie \textbf{ganze} Menge $M$: $D \cup E = M$ \\[3ex]
Für die Ereignisse 
$A := \{1, 2, 3, 4\}$, 
$B := \{2, 4, 6, 8\}$ und 
$C:= \{1, 8\}$ gilt: \\ 
$\surd$ \hspace{1cm} $A$ und $B$ sind unabhängig: $P(A \cap B) =  P(\{2, 4\}) = \frac{1}{4} \eqgood P(A) * P(B)$ \\
$\surd$ \hspace{1cm} $B$ und $C$ sind unabhängig: $P(B \cap C) =  P(\{8\}) = \frac{1}{8} \eqgood P(B) * P(C)$ \\
$\surd$ \hspace{1cm} $A$ und $C$ sind unabhängig: $P(A \cap C) =  P(\{1\}) = \frac{1}{8} \eqgood P(A) * P(C)$ \\
$\times$ \hspace{1cm} $A$ und $B$ und $C$ sind aber nicht unabhängig: \\
\- \hspace{2cm} $P(A \cap B \cap C) =  P(\emptyset) = 0 \eqbad \frac{1}{16} = P(A) * P(B) * P(C)$ \\
\- \dotfill
\\[4ex]
\aufgabe{12}
Ereignis $B$ sind die Farbenblinden, $\overline{B}$ die Leute ohne Sehschwäche.\\
Ereignis $W$ sind die Frauen, $\overline{W}$ die Männer. \\[1ex]
\begin{TAB}(r,1cm,1cm)[5pt]{|r|r|r|r|}{|c|c|c|c|}
KT
&
$B$
&
$\overline{B}$
&
$\sum$
\\
$W$ 
& 
$0,771 \%$ {\scriptsize $(P(B | W) = 1,5 \%)$}
&
$50,629 \%$ {\scriptsize $ = P(W) - P(B \cap W) $}
&
$51,4 \%$ {\scriptsize $ = 1 - P(\overline{W})$}
\\
$\overline{W}$
&
$3,598 \%$ {\scriptsize $(P(B | \overline{W}) = 7 \%)$}
&
$45,002 \%$ {\scriptsize $ = P(\overline{W}) - P(B \cap \overline{W}) $}
&
$48,6 \%$ {\scriptsize gegeben }
\\
$\sum$
&
$4,369 \%$
&
$95,631 \%$
&
$100 \%$
\\
\end{TAB} \\
$P(B) = 4,369 \%$ der Bevölkerung sind farbenblind. \\
Die Ereignisse $B$ und $W$ sind nicht unabhängig, \\
\- \hspace{1cm} weil $0,771 \% = P(B \cap W) \eqbad P(B) * P(W) = 4,369 \% * 51,4 \% = 2,246 \%$ \\
$P(\overline{W} | B) = 3,598 \% / 4,369 \% = 82,35 \%$ der Farbenbilden sind männlich. \\
\- \dotfill
\\[4ex]
\aufgabe{14}
\\[6ex]
\begin{TAB}(r, 2cm, 1ex)[5pt, 15cm, 1cm]{|l|r|r|r|r|r|r|r|}{|c|c|c|}
Kleinste Augensumme
&
0
&
1
&
2
&
3
&
4
&
5
&
6
\\
Wahrscheinlichkeit [1/36]
&
0
&
1
&
3
&
5
&
7
&
9
&
11
\\
kum Verteilungsfunktion [1/36]
&
0
&
1
&
4
&
9
&
16
&
25
&
36
\\
\end{TAB}
\bigskip
\newpage
\aufgabe{13}
\bigskip
Pralinenfabrik Mon Cherie \\[14ex]
\begin{TAB}(r,1cm,1cm)[5pt]{|l|l|}{|t|t|t|}
Ereignis $A$ Praline kommt aus Fabrik A. 
&
Ereigis $B = \overline{A}$ Praline kommt aus Fabrik B. 
\\
Ereignis $K$ Praline enthält weiterhin einen Kern. 
&
Ereignis $\overline{K}$ Praline wurde korrekt entkernt. 
\\
Ereignis $V$ Praline wird verkauft. 
&
Ereignis $\overline{V}$ Praline wird entsorgt. 
\\
\end{TAB} \\[1ex]
$P(K|V) = \frac{P(K \cap V)}{P(V)} = \frac{0,355 \%}{91,397 \%} = 0,388 \% =: p$\\[1ex]
$B(100, p, 0) = (1-p)^{100} = 67,76 \% $ Chance, eine gute Packung zu erwischen. \\[1ex]
\begin{figure}[htbp]
\begin{tikzpicture}  [
    grow                    = right,
    level 1/.style={sibling distance=16em},    
    level 2/.style={sibling distance=7em}, 
    level 3/.style={sibling distance=3em}, 
    level 4/.style={sibling distance=3em}, 
    level distance          = 9em,
    edge from parent/.style = {draw, -latex},
    every node/.style       = {font=\footnotesize},
    sloped
  ]
\node [root] {}
  child { 
    node [env] {$B$}
    child {
      node [env] {$\overline{K}$}
      child {
        node [env] {$\overline{V}$}
        child {
          node [leaf] {$P(B\overline{KV})=0,570 \%$}
          edge from parent node [draw=none] {}           
        }
        edge from parent node [below] {$P(\overline{V} | \overline{K})= 2 \%$}           
      }
      child {
        node [env] {$V$}
        child {
          node [leaf] {$P(B\overline{K}V)=27,930 \%$}
          edge from parent node [draw=none] {}           
        }
        edge from parent node [above] {$P(V | \overline{K})= 98 \%$}           
      }
      edge from parent node [above] {$P(\overline{K} | B)=95 \%$} 
    } %end nK
    child {
      node [env] {$K$}
      child {
        node [env] {$\overline{V}$}
        child {
          node [leaf] {$P(BK\overline{V})=1,425 \%$}
          edge from parent node [draw=none] {}           
        }
        edge from parent node [below] {$P(\overline{V} | K)= 95 \%$}           
      }
      child {
        node [env] {$V$}
        child {
          node [leaf] {$P(BKV)=0,075 \%$}
          edge from parent node [draw=none] {}           
        }
        edge from parent node [above] {$P(V | K)= 5 \%$}           
      }
      edge from parent node [above] {$P(K | B)=5 \%$} 
    }
    edge from parent node [above] {$P(B)=30 \%$} 
  }
  child { 
    node [env] {$A$}
    child {
      node [env] {$\overline{K}$}
      child {
        node [env] {$\overline{V}$}
        child {
          node [leaf] {$P(A\overline{KV})=1,288 \%$}
          edge from parent node [draw=none] {}           
        }
        edge from parent node [below] {$P(\overline{V} | \overline{K})= 2 \%$}           
      }
      child {
        node [env] {$V$}
        child {
          node [leaf] {$P(A\overline{K}V)=63,112 \%$}
          edge from parent node [draw=none] {}           
        }
        edge from parent node [above] {$P(V | \overline{K})= 98 \%$}           
      }
      edge from parent node [above] {$P(\overline{K} | A)=8 \%$} 
    } %end nK
    child {
      node [env] {$K$}
      child {
        node [env] {$\overline{V}$}
        child {
          node [leaf] {$P(AK\overline{V})=5,320 \%$}
          edge from parent node [draw=none] {}           
        }
        edge from parent node [below] {$P(\overline{V} | K)= 95 \%$}           
      }
      child {
        node [env] {$V$}
        child {
          node [leaf] {$P(AKV)=0,280 \%$}
          edge from parent node [draw=none] {}           
        }
        edge from parent node [above] {$P(V | K)= 5 \%$}           
      }
      edge from parent node [above] {$P(K | A)=8 \%$} 
    }
    edge from parent node [above] {$P(A)=70 \%$} 
  };
\end{tikzpicture}
\end{figure}
\end{document}